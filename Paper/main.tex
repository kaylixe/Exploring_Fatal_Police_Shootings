\documentclass[a4paper,12pt]{article}
\usepackage{geometry}
\usepackage[english]{babel}
\usepackage[utf8]{inputenc}
\usepackage{amsmath}
\usepackage{sectsty}
\sectionfont{\centering}
\usepackage{graphicx}
\usepackage[colorinlistoftodos]{todonotes}
\geometry{a4paper,total={170mm,257mm},left=20mm,top=30mm,}

\title{\textbf{Project Proposal\\
\large{Exploring Fatal Police Shootings through Data Visualization}
}}

\author{Felix Kikaya}

\date{\today}
\begin{document}
\maketitle

\begin{abstract}
While crime seems to have gone down in the last few years, the number of police related shootings seem to have been on the rise. High profile cases such the \cite{ferguson},shooting of Michael Brown in Ferguson, Missouri and Laquan McDonald in Chicago, Illinois, have not only drawn so much controversy, but also brought great attention to matters of policing within our society. These cases have poignantly filled majority of newspapers, television stations, as well as a variety of social media platforms and the discussions are endless. With this, movement groups such as Black Lives matter have prompted inquiries about "use of force" and whether there is bias on how police make "shoot or don't shoot" decisions. Using data, a large number of police departments and agencies have found the impetus to analyze and study the effect of race on the officers decisions - while looking at other relevant factors. Studies have shown that there exists a disproportionate number of fatal shootings within minority groups compared to whites, something majority of the communities hasn't been able to has struggle to discern. This project attempts to look at a sample of documented fatal shootings from 2015 to 2017 while paying attention to race and other relevant factors. 
\end{abstract}
\section*{Introduction}
Police brutality has been a contentious subject that is rooted into the history of the United States for many decades. It is not often that such a subject would occur without touching into matters of racial bias and the effect it has on police judgment. Recent advancements in technology such as proliferation of hand-held mobile devices have enabled an almost real-time sharing of information and shown quite a number of cases regarding how policing was being carried out. Another advancement is use of body cameras which police officers have been required to wear while on duty especially when interacting with a suspect. While such technologies where not utilized in the past, data on cases of fatal police shootings were virtually non-existent. Most cases were either not reported nor well documented. 
 
In this era of information science, it is important to utilize the resources that are readily available in our societies and study such cases, using data, and provide a more knowledgeable approach to the issue. Researchers at different agencies such as the Guardian, the vice, and the Washington Post have spearheaded the initiative to collect data from fatal and non-fatal shootings from different police departments across the country. This agencies are still faced with challenges, as some of the data is not made public or some departments cannot provide all the events that lead to the shooting of the suspect. \cite{fatal_force},This project will focus on data specifically collected by the Washington Post. 

The Post \cite{washington_post},began tracking details of each police shootings starting January of 2015, including the circumstances that led up to the shooting. Some of the attributes include race, age, sex, state, city, armed or unarmed e.t.c. They also document whether the officer was equipped with a body camera prior to the shooting. Their databases get updated regularly as they receive information on shootings as they are reported.

\section*{Background Data}
To add onto the shootings dataset, data pulled from four other databases will be used. These will help analyze other relevant factors such as racial population, poverty levels in the community - including median income and level of education, the factors that often determine the magnitude of crime in given communities. \cite{deadly_force} It is expected that when poverty level are high while education is low, crime would be high. This can also have an influence into the police officers judgment of the events leading to the shooting. 
The data used in the project however does not provide more information about the offer involved in the shooting, besides presence of a body camera. Factors such as race, age, sex, experience, are some of the things not reported are some of the things not reported.

\section*{Objectives}
The main objective of this project will be to determine that, if adjusted to population, what is the proportion of different race groups involved in fatal police shootings. This will be determined by the population of the individual race groups in relation to the number of fatal shootings. Other objects are listed as follows;

\begin{itemize}
  \item Determine and compare the number of fatal shootings among different cities. The top ten cities with the highest number of fatal shootings will also be computed. With this, the racial composition of the population will be studied.
  \item The use of body camera in on year to year basis will be analyzed to determine if there has been an increase or a decrease in usage as a result of the recent campaigns.
  \item A distribution of the age of the suspects will be explored while also looking at the gender.
  \item A visualization of median income levels across all the states will be plotted to determine poverty levels as a factor that contributes to crime levels. Change in poverty levels versus percentage of people that have completed high school will be evaluated to determine if there is a direct relationship between education and poverty.
  \item Since mental state of the suspects has been a component in such discussions, cases of mental illness will be determined in each racial group. This will determine how many suspects were mentally ill and also whether they were armed, and what kind of weapon they had.
\end{itemize}
Other objectives will be drawn as the data is being explored.

\bibliographystyle{plain}
\bibliography{fatal}

\end{document}